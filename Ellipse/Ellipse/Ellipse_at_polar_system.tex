% Generated by GrindEQ Word-to-LaTeX 2010 
% ========== UNREGISTERED! ========== Please register! ==========
% LaTeX/AMS-LaTeX

\documentclass{article}

%%% remove comment delimiter ('%') and specify encoding parameter if required,
%%% see TeX documentation for additional info (cp1252-Western,cp1251-Cyrillic)
%\usepackage[cp1252]{inputenc}

%%% remove comment delimiter ('%') and select language if required
%\usepackage[english,spanish]{babel}

\usepackage{amssymb}
\usepackage{amsmath}
\usepackage[dvips]{graphicx}
%%% remove comment delimiter ('%') and specify parameters if required
%\usepackage[dvips]{graphics}

\begin{document}

%%% remove comment delimiter ('%') and select language if required
%\selectlanguage{spanish} 

\noindent A New Strategy to Solving Ellipse around Differential Equation

\noindent 

\noindent Ramin Edjlal

\noindent codesandart@gmail.com

\noindent Keywords: Ellipse, Equation, Around, Accurate, Scope

\noindent 

\noindent 

\noindent Abstract:

\noindent Reaching to ellipse equation in this paper is considered. Differential Equations is a way to reachable equation around of a curve. But because the Ellipse equation is a close curve with symmetry four side we can operate on one side of it and from another because from familiar around curve equation it may we could not get curve around solution we introduce a new strategy in order to find our goal equation. This paper is a way to reaching the exact ellipse equation around. 

\begin{enumerate}
\item  \textbf{Introduction:}
\end{enumerate}

\noindent The polar equation of an ellipse is shown at the below. The $\theta $ in this equation should not be confused with the parameter $\theta $ in the parametric equation. In celestial mechanics, the $\theta $ in the polar equation is called the true anomaly (sometimes denoted by w), while the parameter is called the eccentric anomaly (sometimes denoted by E). The two constants in the polar equation are the semi-latus rectum p and the eccentricity e. The origin is a focus F of the ellipse. There is a second focus F' symmetrically located on the axis. The point P at which r is a minimum is called perihelion in an orbit about the sun, while A is the aphelion. Hence, $2a=\frac{p}{1+e} +\frac{p}{1-e} =\frac{2p}{\mathop{1-e}\nolimits^{2} } $

\noindent Thus:
\[                         (1)                         a=\frac{p}{\mathop{1-e}\nolimits^{2} } \] 


\noindent  Relating p to the semi-major axis a. p is, of course, the radius when $\theta $ = 90${}^\circ$. [1]

\noindent 

\noindent \includegraphics*[width=2.06in, height=1.87in, keepaspectratio=false]{image1.eps}

\noindent Figure \textbf{1}-Polar Equation Of  Ellipse

\noindent Now let c be the distance from the center to either of the foci. Then 
\[                          (2)              c=a-r(e,\theta ,p)\left|\mathop{}\limits_{(e=0)} =\right. a-\frac{p}{1+e} =\frac{ep}{\mathop{1-e}\nolimits^{2} } =ea\] 
With Replacement equation \eqref{GrindEQ__1_} in equation \eqref{GrindEQ__2_} we reached:
\[                          (3)                              c=\frac{ep}{\mathop{1-e}\nolimits^{2} } =ea\] 
That from \eqref{GrindEQ__1_} and \eqref{GrindEQ__3_} we achieve:
\[c=e(\frac{p}{\mathop{1-e}\nolimits^{2} } )=ea\] 
This is the clearest definition of the eccentricity.
\[                                (4)                                 e=\frac{c}{a} .\] 
We see that when e $<$ 1, and e = 0, we reach to a circle. These factors are illustrated in the diagram at the Figure 2. Note especially the right triangle with legs b and c, and hypotenuse a. From this triangle, we can prove that:
\[b=a\sqrt{1-\mathop{e}\nolimits^{2} } \] 

\begin{enumerate}
\item  \textbf{Lemma 1:} In polar coordinate system and at an ellipse we can proof that$b=a\sqrt{1-\mathop{e}\nolimits^{2} } $:
\end{enumerate}

\noindent 
\[\mathop{1-e}\nolimits^{2} =\mathop{1-e}\nolimits^{2} \] 
From equation \eqref{GrindEQ__4_}:
\[\mathop{1-\left(\frac{c}{a} \right)}\nolimits^{2} =\mathop{1-e}\nolimits^{2} \] 
As we know:
\[\mathop{\left(\frac{c}{a} \right)}\nolimits^{2} =\mathop{1-\left(\frac{p}{b} \right)}\nolimits^{2} \] 
With Replacement:
\[                                      (5)                        \mathop{\left(\frac{p}{b} \right)}\nolimits^{2} =\mathop{1-e}\nolimits^{2} \] 
From \eqref{GrindEQ__1_} we have:
\[p=a(\mathop{1-e}\nolimits^{2} )\] 
With replacements:
\[\mathop{\left(\frac{a(\mathop{1-e}\nolimits^{2} )}{b} \right)}\nolimits^{2} =\mathop{1-e}\nolimits^{2} \] 
And with arrangement we reached:
\[                                  (7)                     b=a\sqrt{1-\mathop{e}\nolimits^{2} } \] 
That we achieve to above concept. Below the ellipse is shown the canonical equation of an ellipse, which includes a and b lengths. Two parameters are necessary to specify an ellipse, either a and b or p and e for example

\noindent \includegraphics*[width=2.10in, height=1.77in, keepaspectratio=false]{image2.eps}

\noindent Figure \textbf{2}-The ellispe description

\noindent 

\noindent \includegraphics*[width=2.32in, height=1.59in, keepaspectratio=false]{image3.eps}

\noindent Figure \textbf{3}-Elements

\noindent 

\begin{enumerate}
\item  \textbf{Related Works:}
\end{enumerate}

\noindent Related works about this concept is only approximated works. It means that they estimated the ellipse around at their works. In this paper we try to prepare an exact solution of around ellipse equation. We encountered very tiring problems. 

\begin{enumerate}
\item  \textbf{Hope solution:}
\end{enumerate}

\noindent We can write that:
\[                                (8)                         \mathop{d\tau }\nolimits^{2} =\mathop{dx}\nolimits^{2} +\mathop{dr}\nolimits^{2} \] 
That:
\[                                (9)                      dr=\frac{peSin(\theta )}{\mathop{(1+eCos(\theta ))}\nolimits^{2} } d\theta \] 
And we know that:
\[                               (10)                              dx=rd\theta \] 
And:
\[                               (11)                       dx=\frac{p}{1+eCos(\theta )} d\theta \] 
With replacement:
\[                               (12)                 \mathop{d\tau }\nolimits^{2} =\mathop{\left(\frac{p}{1+eCos(\theta )} \right)}\nolimits^{2} \mathop{d\theta }\nolimits^{2} +\mathop{dr}\nolimits^{2} \] 
If below first integral variable changing solved it will lead to:

\noindent 
\[                               (13)                      1+eCos(\theta )=x\] 
\[\tau d\tau =\frac{-\mathop{p}\nolimits^{2} }{e} \iint \frac{1}{\mathop{x}\nolimits^{2} \sqrt{\mathop{1-(\frac{x-1}{e} )}\nolimits^{2} } } dxd\theta  +rdr\] 
When below variable change is assumed:
\[                                 (14)                      x=1+eCos(u)\] 
We have:
\[                                 (15)          \tau d\tau =\frac{-\mathop{p}\nolimits^{2} }{e} \iint \frac{-e\sin \left(u\right)}{\mathop{\left(1+e\cos \left(u\right)\right)}\nolimits^{2} \left|\sin \left(u\right)\right|} dud\theta  +rdr\] 
The absolute value should be specified the mark.
\[\pi -Arc\cos (\frac{c}{a} )\le \theta \le 0\] 
With cosine:
\[-\frac{c}{a} \le \cos \left(\theta \right)\le 1\] 
With multiply at e:
\[-\frac{ec}{a} \le e\cos \left(\theta \right)\le e\] 
Adding with 1:
\[                           (16)              1-\frac{ec}{a} \le 1+e\cos \left(\theta \right)\le 1+e\] 
From \eqref{GrindEQ__13_} and \eqref{GrindEQ__16_}:
\[                           (17)                        1-\frac{ec}{a} \le x\le 1+e\] 
From \eqref{GrindEQ__14_} and \eqref{GrindEQ__17_}:
\[                           (18)                1-\frac{ec}{a} \le 1+e\cos \left(u\right)\le 1+e\] 
When we return the above steps we see that:
\[0\le u\le \pi -Arc\cos (\frac{c}{a} )\] 
Because the sin(u) in first and second regions is positive thus:

\noindent 
\[                           (19)                              \sin \left(u\right)\ge 0\] 
\includegraphics*[width=3.20in, height=1.93in, keepaspectratio=false]{image4.eps}

\noindent Figure 4-Differencies of a and b vs. u

\noindent The differences changes of length a and b versus u is illustrated in figure 4. The code for reaching to the changes is brought in appendix 1. As we see these changes are linear.

\noindent Thus form \eqref{GrindEQ__19_} we result from \eqref{GrindEQ__15_} we can write:
\[                            (20)        \tau d\tau =\mathop{p}\nolimits^{2} \iint \frac{1}{\mathop{\left(1+e\cos \left(u\right)\right)}\nolimits^{2} } dud\theta  +rdr\] 
From \eqref{GrindEQ__11_} we can write:
\[                            (21)                   d\theta =\frac{1+eCos(\theta )}{p} dx\] 
And from \eqref{GrindEQ__13_} we can write:
\[                            (22)                         d\theta =\frac{x}{p} dx\] 
And From \eqref{GrindEQ__14_} we have:
\[                          (23)               d\theta =\frac{1+eCos(u)}{p} dx\] 
By Derivates \eqref{GrindEQ__14_} we have:
\[                           (24)                   dx=-eSin(u)du\] 
And \eqref{GrindEQ__23_} and \eqref{GrindEQ__24_} we have:
\[                           (25)                   d\theta =-\frac{e(1+eCos(u))}{p} Sin(u)du\] 
From \eqref{GrindEQ__20_} and \eqref{GrindEQ__25_} we have:
\[\tau d\tau =\mathop{p}\nolimits^{2} \iint \frac{1}{\mathop{\left(1+e\cos \left(u\right)\right)}\nolimits^{2} } du\frac{-e(1+eCos(u))}{p} Sin(u)du +rdr\] 
\[\tau d\tau =p\iint \frac{-eSin(u)}{\left(1+e\cos \left(u\right)\right)} dudu +rdr\] 
With integrate:
\[                       (26)          \tau d\tau =p\int Ln\left(1+e\cos \left(u\right)\right) du+rdr\] 


\noindent Depend of \eqref{GrindEQ__34_} and \eqref{GrindEQ__26_}:

\noindent 
\[*\int Ln\left[1+eCos(u)\right] du=-Ln\left[1+eCos(u)\right]u-euCos(u)-\frac{1}{2} Sin(u)+\frac{e}{2} \mathop{Ln[1+eCos(u)]}\nolimits^{2} \] 


\noindent 

\noindent With replacement in \eqref{GrindEQ__26_}
\[\frac{1}{2} \mathop{\tau }\nolimits^{2} =p\left[-Ln\left[1+eCos(u)\right]u-euCos(u)-\frac{1}{2} Sin(u)+\frac{e}{2} \mathop{Ln[1+eCos(u)]}\nolimits^{2} \right]\] 
\[+\frac{1}{2} \mathop{r}\nolimits^{2} +{\rm C}\theta \] 
\textbf{}

\noindent \textbf{}

\noindent \textbf{}

\noindent 

\noindent 

\noindent 

\noindent That C is constant.

\noindent We transfer this formula to section 11

\begin{enumerate}
\item  \textbf{Solving Integral of:}
\[\int Ln\left(a+b\cos \left(u\right)\right) du\] 
\end{enumerate}
If:
\[                          (27)          Ln[a+bCos(u)]=V\] 


\noindent We achieve:
\[                          (28)         du=\frac{-\mathop{E}\nolimits^{V} }{\sqrt{\mathop{b}\nolimits^{2} -\mathop{(a-\mathop{E}\nolimits^{V} )}\nolimits^{2} } } dV\] 
We can write:
\[\int Ln\left[a+bCos(u)\right] du=\int \frac{-\mathop{VE}\nolimits^{V} }{\sqrt{\mathop{b}\nolimits^{2} -\mathop{\left(a-\mathop{E}\nolimits^{V} \right)}\nolimits^{2} } }  dV\] 
And:
\[\int Ln\left[a+bCos(u)\right] du=\int \frac{-\mathop{V(a-a+E}\nolimits^{V} )}{\sqrt{\mathop{b}\nolimits^{2} -\mathop{\left(a-\mathop{E}\nolimits^{V} \right)}\nolimits^{2} } }  dV\] 
By dispreading:
\[\int Ln\left[a+bCos(u)\right] du=-\int \frac{-V\left(\frac{a-E}{b} \right)}{\sqrt{1-\mathop{\left(\frac{a-E}{b} \right)}\nolimits^{2} } }  dV+\int \frac{-Va}{\sqrt{\mathop{b}\nolimits^{2} -\mathop{\left(a-\mathop{E}\nolimits^{V} \right)}\nolimits^{2} } }  dV\] 
That a is constant and is equal to $\mathop{e}\nolimits^{2} $at Ellipse Polar Equation and E is Nipper constant.

\noindent With part of part integral instruction:

\noindent 
\begin{equation} \label{GrindEQ__29_} 
\int Ln\left[a+bCos(u)\right] du=-VArcCos\left(\frac{\mathop{a-E}\nolimits^{V} }{b} \right)+\int ArcCos\left(\frac{\mathop{a-E}\nolimits^{V} }{b} \right) dV+\int \frac{-Va}{\sqrt{\mathop{b}\nolimits^{2} -\mathop{\left(a-\mathop{E}\nolimits^{V} \right)}\nolimits^{2} } }  dV 
\end{equation} 
We know that:
\[                           (30)                  a+bCos(u)=\mathop{E}\nolimits^{V} \] 
With Derivates of \eqref{GrindEQ__3_} we have:
\[                           (31)             -bSin(u)du=\mathop{E}\nolimits^{V} dV\] 
With replacements \eqref{GrindEQ__30_} in \eqref{GrindEQ__31_} we have:
\[                           (32)             dV=-\frac{bSin(u)}{a+bCos(u)} du\] 
From \eqref{GrindEQ__29_} we have:
\[\int Ln\left[a+bCos(u)\right] du=-VArcCos\left((Cos(u)\right)+\int ArcCos\left((Cos(u)\right)dV+ \int \frac{-Va}{\sqrt{\mathop{b}\nolimits^{2} -\mathop{\left(a-\mathop{E}\nolimits^{V} \right)}\nolimits^{2} } }  dV\] 
we have and Simplification: 

\noindent 
\[\int Ln\left[a+bCos(u)\right] du=-Vu-\int \frac{u(bSin(u))}{a+bCos(u)} du +\int \frac{-Va}{\sqrt{\mathop{b}\nolimits^{2} -\mathop{\left(a-\mathop{E}\nolimits^{V} \right)}\nolimits^{2} } }  dV\] 
With part of part of first integral:

\noindent 
\[2\int Ln\left[a+bCos(u)\right] du=\int \frac{-Va}{\sqrt{\mathop{b}\nolimits^{2} -\mathop{\left(a-\mathop{E}\nolimits^{V} \right)}\nolimits^{2} } }  dV\] 
With simplification:
\[\int \frac{V}{\sqrt{\mathop{b}\nolimits^{2} -\mathop{\left(a-\mathop{E}\nolimits^{V} \right)}\nolimits^{2} } }  dV=\frac{b}{2} \mathop{Ln[a+bCos(u)]}\nolimits^{2} \] 
\[\int Ln\left[a+bCos(u)\right] du=-Vu-\int \frac{u(bSin(u))}{a+bCos(u)} du +\frac{b}{2} \mathop{Ln[a+bCos(u)]}\nolimits^{2} \] 
For Integral:
\[bCos(u)=t\] 
\[-bSin(u)du=dt\] 
With part of part integral:
\[\int Ln\left[a+bCos(u)\right] du=-Vu-buCos(u)-\frac{1}{2} Sin(u)+\frac{b}{2} \mathop{Ln[a+bCos(u)]}\nolimits^{2} \] 


\begin{enumerate}
\item  \textbf{Solving Integral of:}
\[\int Ln\left(a+bSin\left(u\right)\right) du\] 
\end{enumerate}
If:
\[                (35)                           Ln[a+bSin(u)]=V\] 


\noindent We achieve:
\[                          (36)         du=\frac{\mathop{E}\nolimits^{V} }{\sqrt{\mathop{b}\nolimits^{2} -\mathop{(a-\mathop{E}\nolimits^{V} )}\nolimits^{2} } } dV\] 
We can write:
\[\int Ln\left[a+bSin(u)\right] du=\int \frac{\mathop{VE}\nolimits^{V} }{\sqrt{\mathop{b}\nolimits^{2} -\mathop{\left(a-\mathop{E}\nolimits^{V} \right)}\nolimits^{2} } }  dV\] 


\noindent And:
\[\int Ln\left[a+bSin(u)\right] du=\int \frac{\mathop{V(a-a+E}\nolimits^{V} )}{\sqrt{\mathop{b}\nolimits^{2} -\mathop{\left(a-\mathop{E}\nolimits^{V} \right)}\nolimits^{2} } }  dV\] 
By dispreading:
\[\int Ln\left[a+bSin(u)\right] du=-\int \frac{V\left(\frac{a-E}{b} \right)}{\sqrt{1-\mathop{\left(\frac{a-E}{b} \right)}\nolimits^{2} } }  dV+\int \frac{Va}{\sqrt{\mathop{b}\nolimits^{2} -\mathop{\left(a-\mathop{E}\nolimits^{V} \right)}\nolimits^{2} } }  dV\] 
We know that:
\[                           (37)                  a+bSin(u)=\mathop{E}\nolimits^{V} \] 
With Derivates of \eqref{GrindEQ__37_} we have:
\[                           (38)             bCos(u)du=\mathop{E}\nolimits^{V} dV\] 
With replacements \eqref{GrindEQ__37_} in \eqref{GrindEQ__38_} we have:
\[                           (39)             dV=\frac{bCos(u)}{a+bSin(u)} du\] 
Like before we have:
\[\int Ln\left[a+bSin(u)\right] du=-VArcSin\left((Sin(u)\right)+\int ArcSin\left((Sin(u)\right)dV+ \int \frac{Va}{\sqrt{\mathop{b}\nolimits^{2} -\mathop{\left(a-\mathop{E}\nolimits^{V} \right)}\nolimits^{2} } }  dV\] 
we have and Simplification: 

\noindent 
\[\int Ln\left[a+bSin(u)\right] du=-Vu+\int \frac{u(bCos(u))}{a+bSin(u)} du +\int \frac{-Va}{\sqrt{\mathop{b}\nolimits^{2} -\mathop{\left(a-\mathop{E}\nolimits^{V} \right)}\nolimits^{2} } }  dV\] 
With part of part of first integral:
\[\int Ln\left[a+bSin(u)\right] du=-Ln[a+bSin(u)]u-\frac{a}{b} u\int Sec(u)du -uLn\left(Cos(u)\right)-\frac{a}{b} \int (\left(-\int Sec(u)du \right) -Ln\left(Cos(u)\right))du             +\int \frac{Va}{\sqrt{\mathop{b}\nolimits^{2} -\mathop{\left(a-\mathop{E}\nolimits^{V} \right)}\nolimits^{2} } }  dV\] 
We know:
\[\int Sec(u)du =Ln\left|Sec\left(u\right)+Tan(u)\right|\] 
With replacement:
\begin{equation} \label{GrindEQ__40_} 
\int Ln\left[a+bSin(u)\right] du=-Ln[a+bSin(u)]u-\frac{a}{b} uLn\left|Sec\left(u\right)+Tan(u)\right|-uLn\left(Cos(u)\right) 
\end{equation} 
\[-\frac{a}{b} \int (\left(-Ln\left|Sec\left(u\right)+Tan(u)\right|\right) -Ln\left(Cos(u)\right))du+\int \frac{Va}{\sqrt{\mathop{b}\nolimits^{2} -\mathop{\left(a-\mathop{E}\nolimits^{V} \right)}\nolimits^{2} } }  dV\] 


\noindent 

\noindent 

\noindent 


\begin{enumerate}
\item  \textbf{Solving Integral of:}
\[\int Ln\left(Cos\left(u\right)\right) du\] 
\end{enumerate}
If:
\[                          (41)          Ln[Cos(u)]=V\] 


\noindent We achieve:
\[                          (42)         du=\frac{-\mathop{E}\nolimits^{V} }{\sqrt{1-\mathop{(\mathop{E}\nolimits^{V} )}\nolimits^{2} } } dV\] 
We can write:
\[\int Ln\left[Cos(u)\right] du=\int \frac{-\mathop{VE}\nolimits^{V} }{\sqrt{1-\mathop{\left(\mathop{E}\nolimits^{V} \right)}\nolimits^{2} } }  dV\] 
And:
\[\int Ln\left[Cos(u)\right] du=\int \frac{-\mathop{V(E}\nolimits^{V} )}{\sqrt{1-\mathop{\left(\mathop{E}\nolimits^{V} \right)}\nolimits^{2} } }  dV\] 
That 'a' is constant and is equal to $\mathop{e}\nolimits^{2} $at Ellipse Polar Equation and E is Nipper constant.

\noindent With part of part integral instruction:

\noindent 
\[                              (43)      \int Ln\left[Cos(u)\right] du=VArcCos\left(\mathop{E}\nolimits^{V} \right)-\int ArcCos\left(\mathop{E}\nolimits^{V} \right) dV\] 
We know that:
\[                           (44)                  Cos(u)=\mathop{E}\nolimits^{V} \] 
With Derivates of \eqref{GrindEQ__44_} we have:
\[                           (45)             -Sin(u)du=\mathop{E}\nolimits^{V} dV\] 
With replacements \eqref{GrindEQ__44_} in \eqref{GrindEQ__45_} we have:
\[                           (46)             dV=-\frac{Cos(u)}{Sin(u)} du\] 
From \eqref{GrindEQ__43_} we have:
\[\int Ln\left[Cos(u)\right] du=VArcCos\left((Cos(u)\right)-\int ArcCos\left((Cos(u)\right)dV \] 
we have and Simplification: 

\noindent 
\[\int Ln\left[Cos(u)\right] du=Vu+\int \frac{u(Cos(u))}{Sin(u)} du \] 
With part of part of first integral:

\noindent 
\[\int Ln\left[Cos(u)\right] du=Ln[Cos(u)]u+u\int Csc(u)du +uLn\left(Sin(u)\right)-\int (\int Csc(u)du) +Ln\left(Sin(u)\right) )du\] 
We know:
\[\int Csc(u)du =-Ln\left|Csc\left(u\right)+Cot(u)\right|\] 
With replacement:
\[\int Ln\left[Cos(u)\right] du=Ln[Cos(u)]u-u\left(Ln\left|Csc\left(u\right)+Cot(u)\right|\right)+uLn\left(Sin(u)\right)\] 
\[-\int (-Ln\left|Csc\left(u\right)+Cot(u)\right| )du-\int ( Ln\left(Sin(u)\right))du\] 
We can write:
\[\left(Ln\left|Csc\left(u\right)+Cot(u)\right|\right)=\left(Ln\left|\frac{1+Cos(u)}{Sin(u)} \right|\right)=\left(Ln\left|1+Cos(u)\right|\right)-\left(Ln\left|Sin(u)\right|\right)\] 
Thus:
\[\int \left(Ln\left|Csc\left(u\right)+Cot(u)\right|\right)du =\int \left(Ln\left|\frac{1+Cos(u)}{Sin(u)} \right|\right)du =\int \left(Ln\left|1+Cos(u)\right|\right)du -\int \left(Ln\left|Sin(u)\right|\right)du \] 
And we can write:
\[\left(Ln\left|Sec\left(u\right)+Tan(u)\right|\right)=\left(Ln\left|\frac{1+Sin(u)}{Cos(u)} \right|\right)=\left(Ln\left|1+Sin(u)\right|\right)-\left(Cos(u)\right)\] 


\noindent Thus:
\[\int \left(Ln\left|Sec\left(u\right)+Tan(u)\right|\right)du =\int \left(Ln\left|\frac{1+Sin(u)}{Cos(u)} \right|\right)du =\int \left(Ln\left|1+Sin(u)\right|\right)du -\int \left(Cos(u)\right)du \] 
Thus:
\[\int Ln\left[Cos(u)\right] du=Ln[Cos(u)]u-u\left(Ln\left|1+Cos(u)\right|\right)-\left(Ln\left|Sin(u)\right|\right)+uLn\left(Sin(u)\right)\] 
\[+\int \left(Ln\left|1+Cos(u)\right|\right)du -\int \left(Ln\left|Sin(u)\right|\right)du -\int ( Ln\left(Sin(u)\right))du\] 
Thus:
\[\int Ln\left[Cos(u)\right] du=Ln[Cos(u)]u-u\left(Ln\left|1+Cos(u)\right|\right)-\left(Ln\left|Sin(u)\right|\right)+uLn\left(Sin(u)\right)\] 
\[+\int \left(Ln\left|1+Cos(u)\right|\right)du -\int \left(Ln\left|Sin(u)\right|\right)du -\left\{\frac{1}{3} \left[2Ln\left(Sin(u)\right)u-u\left(Ln\left|1+Sin(u)\right|\right)-\left(Cos(u)\right)\right. \right. \] 
\[-uLn\left(Cos(u)\right)+\int \left(Ln\left|1+Sin(u)\right|\right)du -\int \left(Cos(u)\right)du +\left\{-u\left(\left(Ln\left|1+Cos(u)\right|\right)-\left(Ln\left|Sin(u)\right|\right)\right)\right. \] 
\[+uLn\left(Sin(u)\right)\left. \left. \left. +\left(\int \left(Ln\left|1+Cos(u)\right|\right)du \right)\right\}\right]\right\}\] 


\begin{enumerate}
\item  \textbf{Solving Integral of:}
\[\int Ln\left(Sin\left(u\right)\right) du\] 
\end{enumerate}
If:
\[                (58)                           Ln[Sin(u)]=V\] 


\noindent We achieve:
\[                  (59)         du=\frac{\mathop{E}\nolimits^{V} }{\sqrt{1-\mathop{(\mathop{E}\nolimits^{V} )}\nolimits^{2} } } dV\] 
We can write:
\[\int Ln\left[Sin(u)\right] du=\int \frac{\mathop{VE}\nolimits^{V} }{\sqrt{1-\mathop{\left(\mathop{E}\nolimits^{V} \right)}\nolimits^{2} } }  dV\] 


\noindent We know that:
\[                           (60)                  Sin(u)=\mathop{E}\nolimits^{V} \] 
With Derivates of \eqref{GrindEQ__36_} we have:
\[                           (61)             Cos(u)du=\mathop{E}\nolimits^{V} dV\] 
With replacements \eqref{GrindEQ__36_} in \eqref{GrindEQ__37_} we have:
\[ (62)                     dV=\frac{Sin(u)}{Cos(u)} du\] 
\textbf{}

\noindent 

\noindent Like before we have:
\[\int Ln\left[Sin(u)\right] du=VArcSin\left((Sin(u)\right)-\int ArcSin\left((Sin(u)\right)dV \] 
we have and Simplification: 

\noindent 
\[\int Ln\left[Sin(u)\right] du=Vu-\int \frac{u(Sin(u))}{Cos(u)} du \] 
With part of part of first integral:
\[\int Ln\left[Sin(u)\right] du=Vu-u\int Sec(u)du -uLn\left(Cos(u)\right)-\int (\left(-\int Sec(u)du \right) -Ln\left(Cos(u)\right))du\] 
We know:
\[\int Sec(u)du =Ln\left|Sec\left(u\right)+Tan(u)\right|\] 
With replacement:
\[         (63)          \int Ln\left[Sin(u)\right] du=Vu-uLn\left|Sec\left(u\right)+Tan(u)\right|-uLn\left(Cos(u)\right)\] 
\[-\int (\left(-Ln\left|Sec\left(u\right)+Tan(u)\right|\right) )du-\int ( -Ln\left(Cos(u)\right))du\] 

With replacements \eqref{GrindEQ__52_} in \eqref{GrindEQ__62_}:
\[         (64)          \int Ln\left[Sin(u)\right] du=Vu-uLn\left|Sec\left(u\right)+Tan(u)\right|-uLn\left(Cos(u)\right)\] 
\[-\int (\left(-Ln\left|Sec\left(u\right)+Tan(u)\right|\right) )du+\left\{Vu-u\left(Ln\left|Csc\left(u\right)+Cot(u)\right|\right)+uLn\left(Sin(u)\right)\right. \] 
\[\left. -\int (-Ln\left|Csc\left(u\right)+Cot(u)\right| )du-\int ( Ln\left(Sin(u)\right))du\right\}\] 
With arrangement:
\[         (65)          2\int Ln\left[Sin(u)\right] du=2Vu-uLn\left|Sec\left(u\right)+Tan(u)\right|-uLn\left(Cos(u)\right)\] 
\[+\int (\left(Ln\left|Sec\left(u\right)+Tan(u)\right|\right) )du+\left\{-u\left(Ln\left|Csc\left(u\right)+Cot(u)\right|\right)+uLn\left(Sin(u)\right)\right. \] 
\[\left. +\int (Ln\left|Csc\left(u\right)+Cot(u)\right| )du\right\}\] 
By Division:
\[\int Ln\left[Sin(u)\right] du=\frac{1}{2} \left[2Vu-uLn\left|Sec\left(u\right)+Tan(u)\right|-uLn\left(Cos(u)\right)\right. \] 
\[+\int (\left(Ln\left|Sec\left(u\right)+Tan(u)\right|\right) )du+\left\{-u\left(Ln\left|Csc\left(u\right)+Cot(u)\right|\right)+uLn\left(Sin(u)\right)\right. \] 
\[\left. \left. +\int (Ln\left|Csc\left(u\right)+Cot(u)\right| )du\right\}\right]\] 
With simplification:
\[\int Ln\left[Sin(u)\right] du=\frac{1}{2} \left[2Vu-uLn\left|Sec\left(u\right)+Tan(u)\right|-uLn\left(Cos(u)\right)\right. \] 
\[+\int (\left(Ln\left|Sec\left(u\right)+Tan(u)\right|\right) )du+\left\{-u\left(Ln\left|Csc\left(u\right)+Cot(u)\right|\right)+uLn\left(Sin(u)\right)\right. \] 
\[\left. \left. +\int (Ln\left|Csc\left(u\right)+Cot(u)\right| )du\right\}\right]\] 
We can write:
\[\left(Ln\left|Csc\left(u\right)+Cot(u)\right|\right)=\left(Ln\left|\frac{1+Cos(u)}{Sin(u)} \right|\right)=\left(Ln\left|1+Cos(u)\right|\right)-\left(Ln\left|Sin(u)\right|\right)\] 
Thus:
\[\int \left(Ln\left|Csc\left(u\right)+Cot(u)\right|\right)du =\int \left(Ln\left|\frac{1+Cos(u)}{Sin(u)} \right|\right)du =\int \left(Ln\left|1+Cos(u)\right|\right)du -\int \left(Ln\left|Sin(u)\right|\right)du \] 
And we can write:
\[\left(Ln\left|Sec\left(u\right)+Tan(u)\right|\right)=\left(Ln\left|\frac{1+Sin(u)}{Cos(u)} \right|\right)=\left(Ln\left|1+Sin(u)\right|\right)-\left(Cos(u)\right)\] 
Thus:
\[\int \left(Ln\left|Sec\left(u\right)+Tan(u)\right|\right)du =\int \left(Ln\left|\frac{1+Sin(u)}{Cos(u)} \right|\right)du =\int \left(Ln\left|1+Sin(u)\right|\right)du -\int \left(Cos(u)\right)du \] 
And we have:
\[\int Ln\left[Sin(u)\right] du=\frac{1}{2} \left[2Vu-u\left(Ln\left|1+Sin(u)\right|\right)-\left(Cos(u)\right)-uLn\left(Cos(u)\right)\right. \] 
\[+\int \left(Ln\left|1+Sin(u)\right|\right)du -\int \left(Cos(u)\right)du +\left\{-u\left(\left(Ln\left|1+Cos(u)\right|\right)-\left(Ln\left|Sin(u)\right|\right)\right)+uLn\left(Sin(u)\right)\right. \] 
\[\left. \left. +\left(\int \left(Ln\left|1+Cos(u)\right|\right)du -\int \left(Ln\left|Sin(u)\right|\right)du \right)\right\}\right]\] 


\noindent And simplification:                  $\frac{3}{2} \int Ln\left[Sin(u)\right] du=\frac{1}{2} \left[2Vu-u\left(Ln\left|1+Sin(u)\right|\right)-\left(Cos(u)\right)-uLn\left(Cos(u)\right)\right. $
\[+\int \left(Ln\left|1+Sin(u)\right|\right)du -\int \left(Cos(u)\right)du +\left\{-u\left(\left(Ln\left|1+Cos(u)\right|\right)-\left(Ln\left|Sin(u)\right|\right)\right)+uLn\left(Sin(u)\right)\right. \] 
\[\left. \left. +\left(\int \left(Ln\left|1+Cos(u)\right|\right)du \right)\right\}\right]\] 


\noindent By Division:\eqref{GrindEQ__66_}                 $\int Ln\left[Sin(u)\right] du=\frac{1}{3} \left[2Vu-u\left(Ln\left|1+Sin(u)\right|\right)-\left(Cos(u)\right)-uLn\left(Cos(u)\right)\right. $
\[+\int \left(Ln\left|1+Sin(u)\right|\right)du -\int \left(Cos(u)\right)du +\left\{-u\left(\left(Ln\left|1+Cos(u)\right|\right)-\left(Ln\left|Sin(u)\right|\right)\right)+uLn\left(Sin(u)\right)\right. \] 
\[\left. \left. +\left(\int \left(Ln\left|1+Cos(u)\right|\right)du \right)\right\}\right]\] 

\begin{enumerate}
\item  \textbf{Solving Integral For  Section 4:}
\end{enumerate}

\noindent From \eqref{GrindEQ__27_} and \eqref{GrindEQ__32_} we have:\textbf{}

\noindent \textbf{}
\[    (67)          \int \frac{V}{\sqrt{\mathop{b}\nolimits^{2} -\mathop{\left(a-\mathop{E}\nolimits^{V} \right)}\nolimits^{2} } }  dV=-\int \frac{Ln[a+bCos(u)]}{\left|Sin(u)\right|}  \frac{\left(bSin(u)\right)}{a+bCos(u)} du\] 
With Simplification:
\[\int \frac{V}{\sqrt{\mathop{b}\nolimits^{2} -\mathop{\left(a-\mathop{E}\nolimits^{V} \right)}\nolimits^{2} } }  dV=-b\int \frac{Ln[a+bCos(u)]}{a+bCos(u)}  du\] 


\noindent If:
\[                    (68)                                   Cos(u)=t\] 
Then:
\[                     (69)                             -Sin(u)du=dt\] 
Then:                   $\int \frac{V}{\sqrt{\mathop{b}\nolimits^{2} -\mathop{\left(a-\mathop{E}\nolimits^{V} \right)}\nolimits^{2} } }  dV=-b\int \frac{-Ln[a+bt]}{(a+bt)}  dt$

\noindent With part of part integral:
\[\int \frac{V}{\sqrt{\mathop{b}\nolimits^{2} -\mathop{\left(a-\mathop{E}\nolimits^{V} \right)}\nolimits^{2} } }  dV=\frac{b}{2} \mathop{Ln[a+bCos(u)]}\nolimits^{2} \] 
From \eqref{GrindEQ__68_}:

\noindent 

\noindent 

\begin{enumerate}
\item  \textbf{Solving Integral For  Section 5:}
\end{enumerate}

\noindent From \eqref{GrindEQ__35_} and \eqref{GrindEQ__39_} we have:\textbf{}

\noindent \textbf{}
\[    (73)          \int \frac{V}{\sqrt{\mathop{b}\nolimits^{2} -\mathop{\left(a-\mathop{E}\nolimits^{V} \right)}\nolimits^{2} } }  dV=\int \frac{Ln[a+bSin(u)]}{\left|Cos(u)\right|}  \frac{\left(a+bSin(u)\right)}{bCos(u)} du\] 
With Simplification:
\[\int \frac{V}{\sqrt{\mathop{b}\nolimits^{2} -\mathop{\left(a-\mathop{E}\nolimits^{V} \right)}\nolimits^{2} } }  dV=b\left(\frac{Cos(u)}{\left|Cos(u)\right|} \right)\int \frac{Ln[a+bSin(u)]}{a+bSin(u)}  du\] 


\noindent If:
\[                    (74)                                   Sin(u)=t\] 
Then:
\[                     (75)                             Cos(u)du=dt\] 
Then:                   $\int \frac{V}{\sqrt{\mathop{b}\nolimits^{2} -\mathop{\left(a-\mathop{E}\nolimits^{V} \right)}\nolimits^{2} } }  dV=\frac{1}{b} \int \frac{Ln[a+bt]}{(a+bt)}  du$

\noindent With part of part integral:
\[\int \frac{V}{\sqrt{\mathop{b}\nolimits^{2} -\mathop{\left(a-\mathop{E}\nolimits^{V} \right)}\nolimits^{2} } }  dV=\frac{1}{b} \left(\frac{Cos(u)}{\left|Cos(u)\right|} \right)\int \frac{(a+bt)Ln[a+bt]}{\mathop{\left(\mathop{1-t}\nolimits^{2} \right)}\nolimits^{{\raise0.7ex\hbox{$ 3 $}\!\mathord{\left/{\vphantom{3 2}}\right.\kern-\nulldelimiterspace}\!\lower0.7ex\hbox{$ 2 $}} } }  du\] 
\[\int \frac{V}{\sqrt{\mathop{b}\nolimits^{2} -\mathop{\left(a-\mathop{E}\nolimits^{V} \right)}\nolimits^{2} } }  dV=\frac{b}{2} \mathop{Ln[a+bSin(u)]}\nolimits^{2} \] 

\begin{enumerate}
\item  \textbf{Reminders:}
\end{enumerate}

\noindent We know some integrals:

\noindent If $\mathop{a}\nolimits^{2} \succ \mathop{b}\nolimits^{2} $:\eqref{GrindEQ__79_}                         $\int \frac{1}{a+bCos(u)} du =\frac{2}{\sqrt{\mathop{a}\nolimits^{2} \_ \mathop{b}\nolimits^{2} } } ArcTan\left[\sqrt{\frac{a-b}{a+b} } Tan\left(\frac{x}{2} \right)\right]$If$\mathop{a}\nolimits^{2} \prec \mathop{b}\nolimits^{2} $:
\begin{equation} \label{GrindEQ__80_} 
\int \frac{1}{a+bCos(u)} du =\frac{1}{\sqrt{\mathop{a}\nolimits^{2} \_ \mathop{b}\nolimits^{2} } } Ln\left|\frac{b+aCos(x)+\sqrt{\mathop{a}\nolimits^{2} \_ \mathop{b}\nolimits^{2} } Sin(x)}{a+bCos(x)} \right| 
\end{equation} 
Else :\eqref{GrindEQ__81_}                                           $\int \frac{1}{1+Cos(u)} du =Tan(\frac{x}{2} )$

\noindent If   $\mathop{a}\nolimits^{2} \succ \mathop{b}\nolimits^{2} $:\eqref{GrindEQ__82_}                      $\int \frac{1}{a+bSin(u)} du =-\frac{2}{\sqrt{\mathop{a}\nolimits^{2} \_ \mathop{b}\nolimits^{2} } } ArcTan\left[\sqrt{\frac{a-b}{a+b} } Tan\left(\frac{\pi }{4} -\frac{x}{2} \right)\right]$If$\mathop{a}\nolimits^{2} \prec \mathop{b}\nolimits^{2} $:
\begin{equation} \label{GrindEQ__83_} 
\int \frac{1}{a+bSin(u)} du =\frac{1}{\sqrt{\mathop{a}\nolimits^{2} \_ \mathop{b}\nolimits^{2} } } Ln\left|\frac{b+aSin(x)+\sqrt{\mathop{a}\nolimits^{2} \_ \mathop{b}\nolimits^{2} } Cos(x)}{a+bSin(x)} \right| 
\end{equation} 
Else: \eqref{GrindEQ__84_}                            $\int \frac{1}{1+Sin(u)} du =-Tan(\frac{\pi }{4} -\frac{x}{2} )$

\noindent 

\noindent 

\begin{enumerate}
\item  \textbf{C founding:}
\end{enumerate}

\noindent If we square and write:
\[\tau =\sqrt{2U(\theta )+2c\theta } \] 
And derive from equation above and arrange we achieve:
\[c=\frac{\tau d\tau }{d\theta } -\mathop{U}\nolimits^{'} \] 
That C =2c.With replacement and arrangement and multiply at$d\theta $:
\[\frac{2(\tau d\tau -\mathop{U}\nolimits^{'} d\theta )}{(\mathop{\tau (\theta )}\nolimits^{2} -2U)} =\frac{d\theta }{\theta } \] 
With integrating and arrangement we achieve:
\[\mathop{\tau (\tau )}\nolimits^{2} =2U+\theta \] 
As result C is equal to 0.5.       

\noindent And From Section 3:

\noindent 
\[\mathop{\tau }\nolimits^{2} =2p\left[-Ln\left[1+eCos(u)\right]u-euCos(u)-\frac{1}{2} Sin(u)+\frac{e}{2} \mathop{Ln[1+eCos(u)]}\nolimits^{2} \right]\] 
\[+\mathop{r}\nolimits^{2} +\theta \] 
\textbf{}

\noindent \textbf{Scope Consideration:}

\noindent The Scope Consideration is derived from symmetry ellipse property. Ellipse is Symmetry of ellipse center and can be divided to four sections.

\noindent 
\[\tau =4\left(\mathop{\left. \theta \right|}\nolimits_{0}^{\frac{\pi }{2} } +\mathop{\left. \theta \right|}\nolimits_{\frac{\pi }{2} }^{\pi -ArcTan(\frac{b}{c} )} \right)\] 

\begin{enumerate}
\item  \textbf{Simulation Consideration: }
\end{enumerate}

\noindent Simulation Consideration leads to very usefulness result on weak personal computers. It seems to be used of more fast and occurrence computers. All we need is simulation of subsidiary Integral and Formula in a Computer Program.

\begin{enumerate}
\item  \textbf{Appendix 1}
\end{enumerate}

\noindent clear all;

\noindent close all;

\noindent a=0:100;

\noindent b=0:100;

\noindent c=sqrt(power(a,2)+power(a,2));

\noindent e=zeros(1,101);

\noindent teta=zeros(1,101);

\noindent x=zeros(1,101);

\noindent u=zeros(1,101);

\noindent for i=1:101

\noindent e(1,i)=c(1,i)/a(1,i);

\noindent teta(1,i)=pi-atan(c(1,i)/a(1,i));

\noindent x(1,i)=1+e(1,i)*cos(teta(1,i));

\noindent u(1,i)=acos((x(1,i)-1)/e(1,i));

\noindent end;

\noindent plot3(a,b,u);

\noindent 

\noindent References:

\noindent [1]- http://mysite.du.edu/\~{}jcalvert/math/ellipse.htm

\noindent [2]-

\noindent Calculus and Analytic Geometry--George B.Thomas,Ross- L.Finney Seventh Edition Addision--Wesely,1988

\noindent [3]-http://in.answers.yahoo.com/question/index?qid=20100620001054AAWDFv9

\noindent [4]-http://www.wolframalpha.com/input/?i=int+\%28+1+\%2F+\%28+3+\%2B+2+cos+x+\%29\%C2\%B2\%29+dx

\noindent 


\end{document}

% == UNREGISTERED! == GrindEQ Word-to-LaTeX 2010 ==

